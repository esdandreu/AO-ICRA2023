% !TeX root = ./Andreu-Gimenez-AO-RoMoCo2024.tex
\documentclass[conference,letterpaper]{src/IEEEtran}
\def\IEEEtitletopspace{18pt}
\IEEEsettextheight{54pt}{54pt}
\IEEEsettextwidth{54pt}{54pt}
\pdfminorversion=4

% Identify funding in the first footnote. If that is unneeded, please comment
% it out.
% \IEEEoverridecommandlockouts

%% Import additional packages and header definitions
% For the LaTex newbies, \input let's you introduce a .tex file as if it were
% part of the current file
%% Bibliography packages recommended for biblatex
\usepackage[T1]{fontenc}
\usepackage[utf8]{inputenc}
\usepackage[english]{babel}
\usepackage{comment}
\usepackage{csquotes}
\usepackage[style=numeric, sorting=ynt]{biblatex}

%% Essential packages
% Colors
\usepackage{xcolor} 
\colorlet{documentLinkColor}{blue}
\colorlet{documentCitationColor}{black!80}
\definecolor{headergray}{rgb}{0.5,0.5,0.5}
% Hyperlinks
\usepackage{hyperref}
\hypersetup{
     colorlinks = true,
     citecolor = documentCitationColor,
     linkcolor = documentLinkColor,
     urlcolor = documentLinkColor,
}
% Diagrams and \foreach
\usepackage{tikz}
\input{src/packages/tkiz_setup.tex}
% Subfigures
\usepackage{caption}
\usepackage[list=true,listformat=simple]{subcaption}
% Table of contents
\usepackage{tocloft}  % Set TOC options.
\setlength{\cftbeforetoctitleskip}{2.0em}
\setlength{\cftaftertoctitleskip}{1.0em}
\setlength{\cftbeforeloftitleskip}{2.0em}  % LOF: List of figures
\setlength{\cftbeforelottitleskip}{2.0em}  % LOT: List of tables
\cftsetpnumwidth{0.75em}  % Prevent overfull hbox on TOC numbers.
% File system
% https://ftp.gust.org.pl/TeX/macros/latex/contrib/currfile/currfile.pdf
\usepackage{currfile}
% Plots
\usepackage{pgfplots}
\pgfplotsset{compat=1.18}
% Good equations
\usepackage{amsfonts}
\usepackage[cmex10]{amsmath}
% Code snippets
\usepackage{listings}
% Better include
\usepackage{newclude}
% Better references
\usepackage[noabbrev,nameinlink]{cleveref}
% Highlight with \hl{...} 
\usepackage{soul}
% Better tables
\usepackage{multirow}

%% Additional packages

% ``float'' algorithms
\usepackage{algorithm}
\usepackage{float}
\newfloat{algorithm}{tbp}{lop}

% Use \singlespace
\usepackage{setspace}
\usepackage{balance}

%% Macros and colour definitions.
\input{src/macros.tex}
\input{src/colors.tex}

%% Developer and debug options
% \overfullrule=5pt % Uncomment this to show a black bar by overfull hboxes.
% \usepackage{layout} % Uncomment and use the \layout command

%% Definitions
\newcommand{\RealSenseDepth}{%
    \href{https://github.com/esdandreu/AO-ICRA2023/raw/master/appendices/recording-devices-specifications/realsense-d435i.pdf}{Intel
        RealSense\texttrademark{} Depth Camera D435i}}
\newcommand{\RealSenseTracking}{%
    \href{https://github.com/esdandreu/AO-ICRA2023/raw/master/appendices/recording-devices-specifications/realsense-t265.pdf}{Intel
        RealSense\texttrademark{} Tracking Camera T265}}
\newcommand{\RODEVideoMicNTG}{%
    \href{https://github.com/esdandreu/AO-ICRA2023/raw/master/appendices/recording-devices-specifications/rode-videmic-ntg.pdf}{RØDE
        VideoMic\texttrademark{} NTG}}
\newcommand{\RODESmartLav}{%
    \href{https://github.com/esdandreu/AO-ICRA2023/raw/master/appendices/recording-devices-specifications/rode-videmic-ntg.pdf}{RØDE
        SmartLav$+$}}
\newcommand{\HPEliteDragonfly}{%
    \href{https://github.com/esdandreu/AO-ICRA2023/raw/master/appendices/recording-devices-specifications/hp-elite-dragonfly.pdf}{HP
        Elite Dragonfly}}

\newcommand{\SRG}{\emph{Ishigami Lab}}

%% Bibliography
% Be careful with reference files list, always avoid whitespaces!
\addbibresource{bibliography/references.bib}

\begin{document}

% ***************************************************************************
%                              Front Matter
% ***************************************************************************

\title{Acoustic Odometry for Wheeled Robots \\ on Loose Sandy Terrain}

\author{
    \IEEEauthorblockN{Andreu Gimenez Bolinches}
    \IEEEauthorblockA{\textit{Graduate School of Science and Technology} \\
        Keio University, Japan \\
        andreu@keio.jp}
    \and
    \IEEEauthorblockN{Genya Ishigami}
    \IEEEauthorblockA{\textit{Graduate School of Science and Technology} \\
        Keio University, Japan \\
        ishigami@mech.keio.ac.jp}
}

\maketitle

\begin{abstract}
    This work proposes a system capable of estimating the longitudinal velocity of
a wheeled robot on loose sandy terrain using only acoustic sensors. These will
likely be present anyway in many future ground mobile robots for human-robot
interaction purposes. The proposed system consists of an audio feature
extraction module, based on gammatone filterbanks, and a prediction module,
based on a convolutional neural network. The proposed system has been tested in
a single wheel test bed with a wheel driving up to speeds of 0.07 m/s with a
wide range of wheel slippage resulting in an average drift of 5 mm/s. A
qualitative evaluation of the proposed system against other sources of odometry
shows that acoustic and visual methods vulnerabilities do not overlap, which
indicates that a system based on acoustic sensors can be a feasible auxiliary
source of odometry. The system is able to make a prediction from a single audio
frame with a duration of 15ms in only 2ms on a user-level commercially
available CPU. Additional experiments with white Gaussian noise show that high
noise power (Signal to Noise Ratio of -10 dB) only affects significantly the
prediction of speeds close to 0ms.
\end{abstract}

\begin{IEEEkeywords}
    Audio-Visual SLAM, Robot Audition, Deep Learning Methods
\end{IEEEkeywords}

% ***************************************************************************
%                              Main Content
% ***************************************************************************
\begin{figure}[t]
    \centering
    \includegraphics[width=\linewidth]{\subdir/system.drawio.png}
    \caption{Proposed system}
    \label{fig:system}
\end{figure}

\include*{\subdir/introduction}
\include*{\subdir/methods}
\include*{\subdir/results}
\include*{\subdir/conclusions}

% ***************************************************************************
%                              Bibliography
% ***************************************************************************
\printbibliography[heading=bibintoc, title={References}]

\end{document}
