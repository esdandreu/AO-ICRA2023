% Define any helpful macros here.

% Parentheses, braces, brackets.
\newcommand*{\parens}[1]{\left( #1 \right)}
\newcommand*{\paren}[1]{\left( #1 \right)}
\newcommand*{\braces}[1]{ \{ #1 \} }
\newcommand*{\brackets}[1]{ \left[ #1 \right] }

% Math stuff.
\DeclareMathOperator*{\argmin}{\arg\!\min}
\DeclareMathOperator*{\argmax}{\arg\!\max}

% Default table and figure dimensions.
\newcommand{\defaultTableWidth}{0.9 \textwidth}
\newcommand{\defaultFigWidth}{0.65}
\newcommand{\defaultAxisWidth}{0.7\textwidth}
\newcommand{\defaultAxisHeight}{0.5\textwidth}

% Footnote for an entire chapter -- no symbol, just text at the bottom.
\newcommand{\chapternote}[1]{{%
  \let\thempfn\relax% Remove footnote number printing mechanism
  \footnotetext[0]{\emph{#1}}% Print footnote text
}}

\newcommand{\tabulargraphics}[2][width=0.25\textwidth]{
  \raisebox{-.5\height}{\includegraphics[#1]{#2}}
}

% Path to a directory named as the current file.
\newcommand{\subdir}{\currfiledir\currfilebase}

% Creates a simple figure taking its filename and looking for it in the
% \subdir. Its label will be the filename without extension. The caption must
% be provided as well.
\makeatletter
\newcommand{\simplefigure}[2]{
    \begin{figure}
        \centering
        \includegraphics[width=\linewidth]{\subdir/#1}
        \caption{#2}
        \filename@parse{#1}
        \label{fig:\filename@base}
    \end{figure}
}
\makeatother

% Left superscripts and subscripts
\newcommand\leftidx[3]{%
  {\vphantom{#2}}#1#2#3%
}