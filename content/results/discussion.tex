\subsection{Discussion} \label{sec:discussion}

One can find in this section insights extracted from the presented results.
First of all, the proposed Acoustic Odometry system proves to be a viable
auxiliary source of odometry for wheeled robots in loose sandy terrain.
\Cref{fig:other-methods} and \cref{fig:VO-challenge} show that audio-based
odometry is more accurate than wheel odometry, and its accuracy can be
comparable to state-of-the-art commercially available visual-based methods and
that it can be more accurate than visual odometry in the presence of dynamic
objects and lightning. 

Secondly, the selected model behavior changes from one device to another.
Devices that output an audio signal with a higher power tend to result in
higher speed predictions, as can be seen in \cref{fig:high-slip}, while the
overall predicted curve is very similar. Meaning that even if the absolute
value of the predicted velocity is not accurate, the model is able to recognize
slippage conditions across different devices. Even with devices not present
during training, as is the case in \cref{fig:other-methods}, where none of the
devices shown in the evaluation were present during training.

Thirdly, white noise affects the performance of the proposed system, as could
be expected. Nevertheless, a considerably large noise power compared to the
signal power does not make the selected model unusable. As it can be seen in
\cref{fig:noise-effect}, where a noise with a power 10 times the one of the
audio signal (SNR -10 dB) only significantly affects the predictions on
speeds close to 0. This indicates that white noise only significantly affects
the prediction of speeds under a threshold determined by the signal-to-noise
power ratio.

Finally, this work can be compared with other odometry methods. On one hand,
only \cite{marchegiani2018a} proposes an audio-based odometry method, claiming
an average absolute error of 0.065 m/s when predicting velocities, which is
comparable to the average \nameref{para:ATE} 0.001 m accumulated over the 66
15 ms long audio frames contained in one second. But this work also shows that
the actual average \nameref{para:RPE} in a 1 second window is 0.005 m. However,
it remains unknown the performance of the proposed system on rotational
movement and under the presence of multiple moving wheels. The comparison
should also be taken with a grain of salt as no common benchmarking data
exists.

On the other hand, the method presented in \cite{Ojeda2006}, where wheel
slippage is identified using the motor current and compensated in the wheel
odometry computation, still outperforms the proposed system while being a
computationally inexpensive method. They demonstrate accumulated drift of up to
1\% of the traveled distance using the same range of longitudinal velocities
under 0.07 m/s, while the proposed system results in an average error of
16\% of the traveled distance across all devices and evaluation recordings.
On the other hand, the authors mention that current-based slippage detection
fails to correctly identify a wheel slipping over rocks instead of sand.
Audio-based methods do have the potential to identify wheel slippage in both
scenarios and \cref{para:recommendations} discusses several improvements that
could be applied to the proposed system in order to improve its performance.