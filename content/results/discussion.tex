\subsection{Discussion} \label{subsec:discussion}

\balance

The selected model behavior changes when it is applied to the output of
different microphones. \Cref{fig:high-slip} shows that the overall predicted
curve is very similar. Meaning that even if the absolute value of the predicted
velocity is not accurate, the model is able to recognize slippage conditions
across different microphones. Even with microphones not present during
training, as is the case in \cref{fig:other-methods}, where none of the
microphones shown in the evaluation were used for training. This indicates that
fine-tuning \cite{TL2016} a model with might significantly increase its
performance on a given microphone. Similarly, using wheel odometry to estimate
the wheel angular speed while using the proposed system to identify the wheel
slippage may improve the performance and generalization.

% Review33: In Discussion section, the authors state "White noise affects the
% performance of the proposed system as could be expected." I fail to
% understand this. Do the author means that white noise will affect the
% performance of the CNN classifier or the Gammatone filter for feature
% extraction. This need to be discussed.

The selected model has been evaluated in presence of synthetic white noise.
It's performance was not significantly affected with signal-to-noise ratio of 0
dB (noise with the same power as the signal). Nevertheless, scenarios with
multiple powered wheels have not been evaluated. 

% Review35: Another vital scenario to consider is when multiple powered
% wheels are operating simultaneously. Robots usually have multiple powered
% revolutions, and it would be good to know the capability of this sensing
% system in the presence of correlated interfering sources.

In \citetitle{marchegiani2018a} \cite{marchegiani2018a}, the authors claim that
their work demonstrates an absolute error of 0.065 m/s and 0.02 rad/s but they
split their training and test set randomly. Their audio frames are generated
with a sliding window of 1 s with 100 ms overlap which means that up to 20\% of
a test audio frame might have been present in test audio frames. Instead, this
work uses a a separate set of recordings for evaluation and conditions never
seen in the training dataset.

The method presented in \citetitle{Ojeda2006} \cite{Ojeda2006}, where wheel
slippage is identified using the motor current and compensated in the wheel
odometry computation, outperforms the proposed system while being a
computationally inexpensive method. They demonstrate accumulated drift of up to
1\% of the traveled distance using the same range of longitudinal velocities
under 0.07 m/s, while the proposed system results in an average error of 16\%
of the traveled distance across all microphones and evaluation recordings.
However, the authors mention that current-based slippage detection fails to
correctly identify a wheel slipping over rocks instead of sand. Audio-based
methods do have the potential to identify wheel slippage in both scenarios.

This work comes with its drawbacks too, However the proposed system has been
evaluated in only one terrain.

% - The one-dimensional scenario can be seen as a simple slip ratio estimation
% problem. 2D pose estimation including orientation must be taken into account
% for Acoustic Odometry to bring value to the real wheeled robot location
% problem.

% - The maximum speed considered of 0.07 m/s is too slow to represent many real
% mobile robot systems.

% - Only one terrain condition has been taken into account.