\subsection{Discussion} \label{subsec:discussion}

\balance

The selected model behavior changes from one device to another. Devices that
output an audio signal with a higher power tend to result in higher speed
predictions, as it can be seen in \cref{fig:high-slip}, while the overall
predicted curve is very similar. Meaning that even if the absolute value of the
predicted velocity is not accurate, the model is able to recognize slippage
conditions across different devices. Even with devices not present during
training, as is the case in \cref{fig:other-methods}, where none of the devices
shown in the evaluation were present during training. This indicates that
fine-tuning \cite{TL2016} a model with might significantly increase its
performance on a given device. Similarly, using wheel odometry to estimate the
wheel angular speed while using the proposed system to identify the wheel
slippage may improve the performance and generalization.

% TODO Review33: In Discussion section, the authors state "White noise affects
% the performance of the proposed system as could be expected." I fail to
% understand this. Do the author means that white noise will affect the
% performance of the CNN classifier or the Gammatone filter for feature
% extraction. This need to be discussed.

White noise affects the performance of the proposed system, as could
be expected. Nevertheless, a considerably large noise power compared to the
signal power does not make the selected model unusable. As it can be seen in
\cref{fig:noise-effect}, where a noise with a power 10 times the one of the
audio signal (SNR -10 dB) only significantly affects the predictions on
speeds close to 0ms. This indicates that white noise only significantly affects
the prediction of speeds under a threshold determined by the signal-to-noise
power ratio.

% TODO Review35: Another vital scenario to consider is when multiple powered
% wheels are operating simultaneously. Robots usually have multiple powered
% revolutions, and it would be good to know the capability of this sensing
% system in the presence of correlated interfering sources.

Finally, this work can be compared with other odometry methods. On one hand,
only \cite{marchegiani2018a} proposes an audio-based odometry method, claiming
an average absolute error of 0.065 m/s when predicting velocities, which is
comparable to the average ATE 0.001 m accumulated over the 66 audio frames
contained in one second of the proposed system. But the proposed system also
shows an average RPE in a 1 second window of 0.005 m, which is significantly
% TODO Review 33: Under Discussion section paragraph 3, please use the IEEE
% format when citing. For example, instead of Marchegiani and Newman, state the
% paper the authors have worked on.
lower that the drift claimed by \citeauthor{marchegiani2018a}. This comparison should
also be taken with a grain of salt as no common benchmarking data exists.
Moreover, it remains unknown the performance of the proposed system on
rotational movement and under the presence of multiple moving wheels.

On the other hand, the method presented in \cite{Ojeda2006}, where wheel
slippage is identified using the motor current and compensated in the wheel
odometry computation, still outperforms the proposed system while being a
computationally inexpensive method. They demonstrate accumulated drift of up to
1\% of the traveled distance using the same range of longitudinal velocities
under 0.07 m/s, while the proposed system results in an average error of
16\% of the traveled distance across all devices and evaluation recordings.
On the other hand, the authors mention that current-based slippage detection
fails to correctly identify a wheel slipping over rocks instead of sand.
Audio-based methods do have the potential to identify wheel slippage in both
scenarios.