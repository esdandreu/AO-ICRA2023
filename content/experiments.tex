\section{Experiments} \label{sec:experiments}

Due to the difficulty in finding publicly available audio odometry datasets,
the problem of robot localization is simplified to a single dimension over a
unique terrain type: A wheel that can only move along a longitudinal axis over
loose sandy terrain. As this scenario can be easily reproduced with the
available \nameref{subsec:experimental-setup}, making it possible to gather a
whole new \nameref{subsec:datasets}. Experiments are conducted varying the
different tuneable parameters in the \nameref{subsec:feature-extraction} module
and \nameref{subsec:prediction-model} and a model is selected based on its
performance and computational cost.

\include*{\subdir/experimental-setup}

\subsection{Datasets} \label{subsec:datasets}

\subsubsection{Training} \label{subsubsec:training-dataset}
used to train and validate the \nameref{subsec:prediction-model}. This work
defines it as a collection of annotated overlapping segments.

The annotated variables are the longitudinal velocity $V_x$, the wheel angular
speed $V_\omega$ and the slip ratio $s$. This work annotates the segments with
a weighted average of the measurements within its duration.

% TODO This phrase is confusing

Segments are collected from a set of training recordings captured under
different driving conditions, which are defined by the combination of the
following controlled variables: \emph{slip ratio} as defined in
\cref{eq:slip-ratio}. Ranging between -0.3 and 0.6. Being free or uncontrolled
slip an additional option; \emph{load} measured in kg added to the base
carriage weight (which is 11.2 kg) taking values of 0 kg, 5 kg and 10 kg;
\emph{wheel angular velocity} ranging from 5 deg/s to 30 deg/s in steps of 5
deg/s; \emph{contact}, whether the wheel is making contact with the ground or
is suspended in the air. A total of 168 different recordings that make a total
of 1 hour for each of the 8 recording devices where captured for the training
dataset.

\subsubsection{Evaluation} \label{subsubsec:evaluatio-dataset}
used to evaluate the performance of the system under more realistic and
challenging conditions unseen during training like high slippage induced by an
external force. Composed of 7 recordings with a total duration of 17 minutes.

\subsection{Selected model} \label{subsec:selected-model} 

% TODO Reveiw33: Why is the range between [50,80000]? 80KHz is beyond audible
% frequency range. Are the microphone able to capture at this frequency range?
% What microphones are being used?

This work selects a model trained on a dataset of segments of 50 frames, each
of them spans over 15 milliseconds and has 64 features from a single
\nameref{subsubsec:gammatone-filterbank} extractor. The extractor is applied on
the average of the audio signal channels with a frequency range of [50, 80000]
Hz and features are represented on the Bel scale. The model is trained with
data from all available recordings but only devices 3, 4, 6 and 8 as listed in
\cref{table:wheeltestbed-setup-2-devices}. Training data is augmented with
added random SNR noise. The model task is to classify the longitudinal velocity
given a set of 28 different linearly distributed longitudinal velocity ranges
taking into account the order of the classes, ordinal classification, as
introduced in \cite{ordclass2006}. The selected model makes use of a batch
normalization layer and has a size of 51.5MB. 