\section{Experiments} \label{sec:experiments}

Due to the difficulty in finding publicly available audio odometry datasets,
the problem of robot localization is simplified to a single dimension over a
unique terrain type: A wheel that can only move along a longitudinal axis over
loose sandy terrain. This scenario can be easily reproduced with the available
\nameref{subsec:experimental-setup}, making it possible to gather new
\nameref{subsec:datasets}. 

\include*{\subdir/experimental-setup}

\subsection{Datasets} \label{subsec:datasets}

\subsubsection{Training} \label{subsubsec:training-dataset} A collection of
overlapping audio frames used to train and test the
\nameref{subsec:prediction-model}. Audio frames are annotated with longitudinal
velocity, the wheel angular speed and the slip ratio with a weighted average of
the measurements within the frame duration.

Audio frames are extracted from test recordings where the wheel test bed is
used to control different variables as the slip ratio and the wheel angular
velocity. 168 different recordings that make a total of 1 hour for each of the
6 microphones where captured for the training dataset.  

\subsubsection{Evaluation} \label{subsubsec:evaluatio-dataset} 7 recordings
with a total duration of 17 minutes used to evaluate the performance of the
system under more realistic and challenging conditions. Unlike in the test
dataset, where the longitudinal position was controlled by the wheel test bed,
the carriage unit is allowed free longitudinal movement and only the wheel
drive motor is active. High wheel slippage is induced by external forces and
a slopped terrain.


\subsection{Selected model} \label{subsec:selected-model} 

% Review29 In methods, this paper shows very limited innovation in feature
% extraction and prediction model. The design and implementation of the CNN is
% unclear because the elaboration of Figure 3 is insufficient. The direct
% output of the prediction model is also unclear.

This work selects a model trained on a dataset of segments of 50 frames, each
of them spans over 15 milliseconds and has 64 features from a single Gammatone
filterbank extractor. The extractor is applied on the average of the audio
signal channels with a frequency range of [50, 8000] Hz and features are
represented on the Bel scale. The model is trained with data from a subset of
microphones (1, 2, 4 and 6 as shown in \cref{fig:wheeltestbed-setup-2}).
Training data is augmented with added random SNR noise. The model classifies
the longitudinal velocity given a set of 28 different linearly distributed
longitudinal velocity ranges taking into account the order of the classes,
ordinal classification, as introduced in \cite{ordclass2006}. Therefore the
output of the model is a vector of probabilities that can be decoded into a
class position by making use of a ranking rule. The loss is computed with the
mean square error between the target class position and the predicted class
position.