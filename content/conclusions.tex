\section{Conclusions} \label{chap:conclusions}

This work demonstrates that it is feasible to estimate odometry from acoustic
data with a computationally inexpensive system. Which can add robustness to the
localization system without incurring in significant costs.

The proposed system is capable of estimating the longitudinal velocity of a
wheeled robot on loose sandy terrain using gammatone-based features extracted
from robot ego-noise data and an ordinal classification model implemented as a
convolutional neural network. Said system has been trained and evaluated in a
new multi-modal dataset collected using a single wheel testbed and several
microphones and cameras. The evaluation contains high wheel slippage scenarios
where the proposed system successfully identifies when the wheel slips. The
proposed model is also evaluated in the presence of white noise and
demonstrates that it still can successfully predict longitudinal velocities in
the presence of high noise power. Moreover, the system is compared against
wheel odometry and a commercially available visual simultaneous localization
and mapping system satisfactorily.

% Finally, in recent years, single-robot simultaneous localization and mapping
% research is steadily moving toward systems that can build metric-semantic maps.
% Acoustic odometry could be combined with terrain classification in order to
% provide both, an auxiliary source of odometry and semantic information for a
% simultaneous localization and mapping system. Similarly, a system that not only
% estimates motion based on ego-noise but also is able to identify and subtract
% said ego-noise from the audio signal would be useful to improve the performance
% of speech recognition in collaborative ground robots. 