\section{Conclusions} \label{chap:conclusions}

% Review 33: You should always state quantitatively the
% performance of your method against state-of-the-art. You state at the end of
% your concluding statement that your method performance satisfactorily then
% what research question were you trying to answer in this paper?

The proposed system is capable of estimating the longitudinal velocity of a
wheeled robot on loose sandy terrain using gammatone-based features extracted
from robot ego-noise data and an ordinal classification model implemented as a
convolutional neural network. Said system has been trained and evaluated in a
new multi-modal dataset collected using a single wheel testbed and several
microphones and cameras. The evaluation contains high wheel slippage scenarios
where the proposed system achieves an average drift of 5 mm per second with
longitudinal velocities under 0.07 m/s. The proposed system runs 7.5 times
faster than real time on a user-level hardware.

% Comments to author: The claim that the proposed method estimates robot
% odometry is a bit exaggerated: it really assesses horizontal velocity in
% order to correct wheel slippage.

However, claiming that the proposed method estimates robot odometry in its
current form would be an exaggeration. The experiments presented in this paper
correspond to an oversimplified scenario. Nevertheless, this work demonstrates
that it is feasible to estimate longitudinal velocity of a wheeled robot on
loose sandy terrain from acoustic data with a computationally inexpensive
system. Which indicates that an acoustic odometry system can be a feasible
auxiliary source of odometry and add robustness to the localization system
without incurring in significant costs. 



% Comments to author: The experiments do not take into account sufficiently
% realistic scenarios (e.g. too slow speed, not unstructured environment) and
% do not compare with real state-of-the-art techniques (Intel RealSense system
% is not visual SLAM system)

Finally, in recent years, single-robot simultaneous localization and mapping
research is steadily moving toward systems that can build metric-semantic maps.
Acoustic odometry could be combined with terrain classification in order to
provide both, an auxiliary source of odometry and semantic information for a
simultaneous localization and mapping system. Similarly, a system that not only
estimates motion based on ego-noise but also is able to identify and subtract
said ego-noise from the audio signal would be useful to improve the performance
of speech recognition in collaborative ground robots. 