\subsection{Prediction Model} \label{subsec:prediction-model}

Consecutive feature frames from the \nameref{subsec:feature-extraction} module
are concatenated in order to form a \emph{segment}. This segment is then fed to
the prediction model. Which will output a motion estimation. Segments 

This work proposes a shallow Convolutional Neural Network \cite{FukushimaCNN}
composed by two convolution layers, each of them followed by a max pooling
layer, and two fully connected layers, each of them preceded by a dropout, as
shown in \cref{fig:model-arch-cnn}. 

\begin{figure}
    \centering
    \includegraphics[width=\linewidth]{\subdir/CNN.drawio.png}
    \caption{Convolutional Neural Network architecture used in this work.}
    \label{fig:model-arch-cnn}
\end{figure}


% The input dimensions depend on the \nameref{subsec:feature-extraction}
% parameters, namely number of features per frame, number of extractors and
% number of frames per segment. Different layer sizes are evaluated as well,
% defined in \cref{table:model-arch-sizes}. The output of the last fully
% connected layer depends on the \nameref{subsec:model-task}.

% \subsubsection{Task} \label{subsec:model-task}

% One can find here a description of the different tasks implemented and tested
% in models. Tasks define the goal of the model and the way its loss is computed.

% \paragraph{Classification} \label{para:model-task-class} Consists in
% classifying the longitudinal velocity given a set of possibilities. The
% different classes are ranges of longitudinal velocities, being these ranges a
% hyperparameter of the model. Cross entropy loss is computed between the
% predicted class probabilities and the class corresponding to the target
% longitudinal velocity and the predicted class. The output of the model is
% therefore a vector of probabilities corresponding to each class.

% \paragraph{Ordinal classification} \label{para:model-task-ord-class} Consists
% in classifying the longitudinal velocity given a set of possibilities like in
% \nameref{para:model-task-class}, with different classes being ranges of
% longitudinal velocities. But the order of the class matters. This method was
% introduced in \cite{ordclass2006}, where standard classification algorithms are
% extended to make use of the order of the classes. The output of this model is a
% vector of binary values that can be decoded into a class position by making use
% of a ranking rule. The loss is computed with the mean square error between the
% target class position and the predicted class position. 

% \subsubsection{Architecture} \label{subsec:model-architecture}

% This section describes the different model architectures implemented and
% evaluated. A common point of them all is simplicity. It is out of the scope of
% this work to find an optimal architecture for acoustic odometry. But it is
% interesting to evaluate different simple options.

% \paragraph{CNN with normalized input} \label{para:model-arch-norm-cnn} This
% architecture is identical to the \nameref{para:model-arch-cnn} except for the
% fact that it contains a batch normalization layer \cite{batchnorm2015} as shown
% in \cref{fig:model-arch-cnn}.