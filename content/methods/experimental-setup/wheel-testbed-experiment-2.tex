\subsubsection{Wheel Testbed Experiment 2}
\label{subsec:wheel-testbed-experiment-2}

On one hand, the number of sensors used sums up to a total of 5 external
microphones plus the sensor workstation internal microphone array and two
cameras, one of them a tracking camera with built-in visual-based simultaneous
localization and mapping (Visual SLAM). The final list of devices can be found
in \cref{table:wheeltestbed-setup-2-devices} while their positioning is shown
in \cref{fig:wheeltestbed-setup-2}. Their positioning takes into account ideal
positions for a microphone attempting to capture the sound generated by the
wheel interaction with the terrain (devices 4 and 3), feasible positions for a
microphone in a mobile robot (devices 5, 6, and 7), and a reasonable position
for a microphone dedicated for human-robot interaction (device 8).


\begin{figure}
    \begin{subfigure}{.57\linewidth}
        \centering
        \input{\subdir/wheeltestbed-setup-lateral.tkiz}
        \caption{Lateral view}
        \label{fig:wheeltestbed-setup-2-lateral}
    \end{subfigure}
    \hfill
    \begin{subfigure}{.40\linewidth}
        \begin{subfigure}{\linewidth}
            \centering
            \input{\subdir/wheeltestbed-setup-back.tkiz}
            \caption{Back view}
            \label{fig:wheeltestbed-setup-2-back}
        \end{subfigure}
        \bigskip
        
        \begin{subfigure}{\linewidth}
            \centering
            \input{\subdir/wheeltestbed-setup-front.tkiz}
            \caption{Front view}
            \label{fig:wheeltestbed-setup-2-front}
        \end{subfigure}
    \end{subfigure}
    \caption[\nameref{subsec:wheel-testbed-experiment-2} setup]{
        \nameref{subsec:wheel-testbed-experiment-2} setup: \emph{a)} shows a
        lateral view of the wheel testbed carriage while being weighted.
        \emph{b)} shows a back view of the wheel and sensor setup while
        \emph{c)} shows a front view of it.}
    \label{fig:wheeltestbed-setup-2}
\end{figure}

\begin{table}
    \centering
    \begin{tabular}{| c | c|}
        \hline
        Index & Device                                        \\ \hline \hline
        1     & \RealSenseDepth                               \\ \hline
        2     & \RealSenseTracking                            \\ \hline
        3     & \RODEVideoMicNTG{} front                      \\ \hline
        4     & \RODEVideoMicNTG{} back                       \\ \hline
        5     & \RODEVideoMicNTG{} top                        \\ \hline
        6     & \RODESmartLav{} wheel axis                    \\ \hline
        7     & \RODESmartLav{} top                           \\ \hline
        8     & \HPEliteDragonfly{} built in microphone array \\ \hline
    \end{tabular}
    \caption{Devices used in the \nameref{subsec:wheel-testbed-experiment-2}}
    \label{table:wheeltestbed-setup-2-devices}
\end{table}

On the other hand, some maintenance and reparations were applied to the wheel
testbed. The ball screw shaft was reassembled with tighter screws and grease
was applied to the ball bearings. Additionally, the cables were rearranged in a
way that they do not appear in the field of view of the camera during the
recording.

The experiment is composed of different recordings on the wheel testbed.
Recordings are performed under different driving conditions, which are defined
by the combination of the following controlled variables: \emph{slip ratio} as
defined in \cref{eq:slip-ratio}. Ranging between -0.3 and 0.6. Being free
or uncontrolled slip an additional option; \emph{load} measured in kg added
to the base carriage weight (which is 11.2 kg) taking values of 0 kg, 5
kg and 10 kg; \emph{wheel angular velocity} ranging from 5 deg/s to 30
deg/s in steps of 5 deg/s; \emph{contact}, whether the wheel is making
contact with the ground or is suspended in the air. 

Combinations of said variables resulted in 168 different recordings that make
a total of 1 hour for each of the 8 recording devices. It took approximately 13
working hours to perform. The data gathered is used to build the different
datasets shown in \cref{sec:datasets}.