\subsubsection{Wheel Testbed Experiment 1}
\label{subsec:wheel-testbed-experiment-1}
This experiment improved the sensor setup, synchronization, and control with
respect to the \nameref{subsec:wheel-testbed-experiment-0}.

Firstly, the number of sensors was increased to a total of 4 microphones and 1
stereo camera with IMU. Their final positions are shown in
\cref{fig:wheeltestbed-setup-1} and hyperlinks to their specifications data
sheets can be found in \cref{table:wheeltestbed-setup-1-devices}.

\begin{figure}
    \begin{subfigure}{\textwidth}
        \centering
        \input{\subdir/wheeltestbed-setup-lateral.tkiz}
        \caption{Lateral view}
        \label{fig:wheeltestbed-setup-1-lateral}
    \end{subfigure}
    \bigskip
    
    \begin{subfigure}{.45\textwidth}
        \centering
        \input{\subdir/wheeltestbed-setup-back.tkiz}
        \caption{Back view}
        \label{fig:wheeltestbed-setup-1-back}
    \end{subfigure}
    \hfill
    \begin{subfigure}{.45\textwidth}
        \centering
        \input{\subdir/wheeltestbed-setup-front.tkiz}
        \caption{Front view}
        \label{fig:wheeltestbed-setup-1-front}
    \end{subfigure}
    \caption[\nameref{subsec:wheel-testbed-experiment-1} setup]{
        \nameref{subsec:wheel-testbed-experiment-1} setup: \emph{a)} shows a
        lateral view of the whole wheel testbed. Special attention should be
        paid to the fact that the sensors workstation and the testbed
        workstation are independent and are not physically connected. \emph{b)}
        shows a back view of the wheel and sensor setup while \emph{c)} shows a
        front view of it. }
    \label{fig:wheeltestbed-setup-1}
\end{figure}


\begin{table}
    \centering
    \begin{tabular}{| c | c|}
        \hline
        Index & Device                     \\ \hline \hline
        1     & \RealSenseDepth            \\ \hline
        2     & \RODEVideoMicNTG{} front   \\ \hline
        3     & \RODEVideoMicNTG{} back    \\ \hline
        4     & \RODESmartLav{} wheel axis \\ \hline
        5     & \RODESmartLav{} top        \\ \hline
    \end{tabular}
    \caption{Devices used in the \nameref{subsec:wheel-testbed-experiment-1}}
    \label{table:wheeltestbed-setup-1-devices}
\end{table}

Secondly, the synchronization of the sensors was improved by developing a
device-independent framework for synchronized recording:
\href{https://github.com/AcousticOdometry/python-recorder}{\texttt{python-recorder}}.
It gives researchers an easy-to-use command line application that allows them
to configure, test, and use the available sensors in the host machine while
providing intuitive mechanisms to synchronize the recording devices with the
specific event that one wants to record. More information about it can be found
in the \href{https://pypi.org/project/python-recorder/}{Python Package Index}.
A key concept of this framework is that the wheel testbed control is executed
in one machine (\cref{fig:wheeltestbed-setup-1-lateral} Testbed Workstation),
while a different machine is used to control the recording and store the data
(\cref{fig:wheeltestbed-setup-1-lateral} Sensor Workstation) without giving up
the synchronization between the wheel movement and the recording. Although it
may seem a more complex setup than with a single computer, the hardware setup
is easier this way as it simplifies the cabling.

Thirdly, the control software for the Wheel Testbed from \SRG{} was improved by
reworking the timing mechanisms for the control algorithm with modern
\texttt{C++} timing features. Additionally, a new experiment script was created
to automate significantly the process. It synchronizes the recording with the
wheel movement using \texttt{python-recorder}, gives the output data a
standardized file name, and provides instructions and reminders on the command
line for a successful test. It was developed by having in mind long recording
sessions where human errors are common.

Unfortunately, several hiccups were suffered during the execution of this
experiment. A loose screw in the motor mounting started to collide with the
motor axle and ball screw coupling. The whole longitudinal movement contraption
had to be disassembled and screws tightened. Additionally, device number 5 (a
small Microphone mounted on top of the wheel) failed to record when device
number 1 (An RGB-D camera with IMU) was recording. An issue that was not
identified before several experiment sessions since the microphone worked as
expected whenever the camera was not activated, which was the usual way of
testing the setup.

In any case, this experiment was useful for gathering a significant amount of
data used to train a prototype model and to set up a better experiment:
\nameref{subsec:wheel-testbed-experiment-2}. The experiment gathered 34 minutes
of recording for each of the working devices (3 microphones, 1 RGB-D camera)
and it took approximately 9 working hours to perform.