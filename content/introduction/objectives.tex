\subsection{Objectives and contributions} \label{subsec:objectives}

The goal of this work is to study the feasibility of an audio-based odometry
system, since it has the potential of providing inexpensive robust robot
localization in the future, as mentioned in \cref{subsec:motivation}. More
concretely, the objectives of this thesis are:

\begin{itemize}
    \item To design a system capable of estimating the longitudinal velocity of
          a wheel driving over loose sandy terrain using only acoustic information.
          
    \item To quantitatively evaluate the performance of acoustic odometry
          against other odometry systems.
          
    \item To qualitatively evaluate the feasibility of using audio-based
          odometry in a real-world situation.
\end{itemize}

Due to the difficulty in finding publicly available audio odometry datasets,
the problem of robot localization is simplified to a single dimension over a
unique terrain type: A wheel that can only move along a longitudinal axis over
loose sandy terrain. As this scenario can be easily reproduced with the
available experimental equipment, making it possible to gather a whole new
dataset.

\Cref{chap:methods} describes the proposed system, including
\cref{sec:experimental-setup}, where it is described how the data used in this
work was gathered, \cref{sec:datasets}, which explains how that data is
processed and \cref{sec:models}, where one can find details about how processed
data is used to predict the longitudinal velocity of a wheeled robot.
\Cref{chap:results} objectively states its evaluation results with a subjective
discussion in \cref{sec:discussion}. Finally, \cref{chap:conclusions}
summarizes the most important insights from this research as well as
recommendations for future research on the field of acoustic odometry.
