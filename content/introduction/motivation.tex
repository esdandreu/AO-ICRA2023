\subsection{Motivation} \label{subsec:motivation}

This work intends to minimize the price of robust ground robot localization in
unstructured environments by using sensors that are relatively inexpensive,
imply a low overhead, and might be needed for other functionalities of the
robot: acoustic odometry. Many ground mobile robots will be equipped with sound
sensors for the purpose of human-robot interaction \cite{VoiceForklift}, at the
same time, audio signals have been extensively studied, resulting in very
efficient methods for their processing.

Increasing the robustness of robot localization in unstructured environments
without significantly increasing its price would make this technology more
accessible. Which complies with the 9th and 10th Sustainable Development Goals:
Promotes innovation without increasing the inequalities between small and big
producers \cite{SDG}.

At the same time, this work focuses on wheeled robots on loose sandy terrain.
Although strides have been made into exotic forms of legged robots, wheels or
tracks still form the basis for robot locomotion, although strides
\cite{Sanchez09}. Wheeled mobile systems are useful for practical applications
compared with legged systems because of the simplicity of the mechanisms and
control systems and efficiency in energy consumption \cite{Masayoshi2006}.
However, wheeled systems' performance depends on the traction between the
wheels and the ground. If there is not enough traction, the wheel will slip and
the efficiency will decrease. Traction is a special concern when the robot is
expected to move over granular non-cohesive loose soil. Which is the case for
planetary missions \cite{Amar2007}, construction site applications, and
agricultural robots, among others.