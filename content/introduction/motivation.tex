\subsection{Motivation} \label{subsec:motivation}

One way to minimize the price of robust ground robot localization in
unstructured environments is using sensors that are relatively inexpensive,
imply a low overhead, and might be needed for other functionalities of the
robot. Many ground mobile robots will be equipped with sound sensors for the
purpose of human-robot interaction \cite{VoiceForklift}, at the same time,
audio signals have been extensively studied, resulting in very efficient
methods for their processing. Therefore, this work intends to asses the
feasibility of acoustic odometry as an auxiliary source of odometry.

This work focuses on wheeled robots on loose sandy terrain. Although strides
have been made into exotic forms of legged robots, wheels or tracks still form
the basis for robot locomotion \cite{Sanchez09}. Wheeled mobile systems are
useful for practical applications compared to legged systems because of the
simplicity of the mechanisms and control systems and efficiency in energy
consumption \cite{Masayoshi2006}. However, wheeled systems' performance depends
on the traction between the wheels and the ground. If there is not enough
traction, the wheel will slip and the efficiency will decrease. Traction is a
special concern when the robot is expected to move over granular non-cohesive
loose soil. Planetary exploration, construction and agriculture are some of the
applications where robots operate on sandy terrain.