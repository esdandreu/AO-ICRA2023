\section{Results} \label{sec:results}

\Cref{table:results-selected-model} shows the average performance of the
selected model across all seven evaluation recordings and devices. Performance
is measured with commonly used odometry and simultaneous localization and
mapping metrics as described in \cite{Measuring2019}. \hl{Additionally APE}
\Cref{fig:high-slip} shows the its results on an evaluation recording
characterized by high slippage driving conditions using two different devices.

\begin{table}
    \centering
    \begin{tabular}{|c|c|c|}
        \hline
                          & Mean ATE [m] & Mean RPE [m/s] \\ \hline
        Acoustic Odometry & $1.02e-03$   & $5.10e-03$     \\
        \hline
    \end{tabular}
    \caption[Selected model average performance across evaluation recordings
        and devices]{Selected model average performance across all evaluation
        recordings and all devices. Average Trajectory Error (ATE) is computed
        between frames with a duration of 15 ms and Relative Pose Error (RPE)
        is computed using time windows 1s long.}
    \label{table:results-selected-model}
\end{table}


\begin{figure}
    \centering
    \includegraphics[width=0.95\linewidth]{\subdir/high-slip.png}
    \caption[Selected model on high slippage conditions]{Selected model
        evaluated on a recording characterized by high slippage conditions
        using two different devices.\hl{WIP}}
    \label{fig:high-slip}
\end{figure}

\subsection{Noise} One can see in \cref{fig:noise-effect} the behavior of the
selected model against white noise. Generally, white noise increases the value
of the estimated motion. Signal-to-Noise ratios of -20 dB impede the model to
recognize situations where the wheel is not moving.

\begin{figure}
    \centering
    \includegraphics[width=\linewidth]{\subdir/noise-effect.png}
    \caption[Selected model with white noise]{Selected model evaluated on a
        recording with added white noise with varying signal-to-noise Ratio
        values.\hl{WIP}}
    \label{fig:noise-effect}
\end{figure}

\subsection{Computational cost} Tests with the selected model on a CPU show that
the time to compute the feature extraction is a 40\% of the total time to
process a new frame, which is 2.03 ms on an Intel\textregistered{}
Core\texttrademark{} i7-9750H CPU at 2.60GHz. The prediction time takes up a
55\% of the total time while the remaining 5\% corresponds to loading the new
frame into memory The system is able to run 7.5 times faster than real time on
this particular hardware. Using hardware acceleration with CUDA the total time
to process a new frame is slightly lower: 1.93 ms on a NVIDIA GeForce RTX
2060 GPU.

\subsection{Compared with other models} The selected model is evaluated against
two other odometry methods: Wheel odometry computed from the ground truth wheel
angular speed; Visual SLAM using Intel\textregistered{}
RealSense\texttrademark{} Tracking Camera T265, which is based on visual
odometry. \Cref{fig:other-methods} is the evaluation in an undisturbed scenario
for all methods, no acoustic noise and no lightning changes or dynamic objects.
It shows that the selected model can perform with an accuracy comparable to
state-of-the-art commercial visual-based methods in this simplified scenario.
\Cref{fig:VO-challenge} instead corresponds to a scenario that is challenging
for visual odometry. Dynamic objects are moved in the camera field of view and
lightning conditions are changed during the recording. One can see in this
evaluation that the vulnerabilities of audio-based odometry and visual-based
odometry do not overlap. Which makes Acoustic Odometry more accurate in certain
scenarios where the visual odometry vulnerabilities are exploited.

\begin{figure}
    \centering
    \includegraphics[width=\linewidth]{\subdir/other-methods.png}
    \caption{Selected model evaluated
        evaluated against Visual SLAM (based on visual odometry) and wheel
        odometry.}
    \label{fig:other-methods}
\end{figure}

\begin{figure}
    \centering
    \includegraphics[width=\linewidth]{\subdir/VO-challenge.png}
    \caption{
        Selected model evaluated against Visual SLAM (based on visual odometry)
        in a scenario where visual odometry vulnerabilities are exploited:
        Dynamic objects are moved in the camera field of view and lightning
        conditions are changed during the recording.}
    \label{fig:VO-challenge}
\end{figure}

\include*{\subdir/discussion}