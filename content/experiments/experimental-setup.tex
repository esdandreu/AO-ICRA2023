\subsection{Experimental Setup} \label{subsec:experimental-setup}

% Comments to author: Most of the paper is devoted to the presentation of
% experimental setup, the collection of the dataset and the training of the
% CNN. 

This work uses a single wheel testbed, composed of a carriage unit attached to
a frame over a soil bin shown in \cref{fig:wheeltestbed-setup-2}. It is 3500 mm
in length, 600 mm in width, and 1200 mm in height. The soil bin is filled with
sandy soil with a depth of 200 mm. The surface of the sand can be leveled or
sloped before each test run using a leveling apparatus attached to the sandbox.

The carriage unit can move in the longitudinal direction at an arbitrary
velocity using the ball screw while the unit freely moves in its vertical
direction. The wheel is placed at the bottom of the carriage wheel with an
independent traction motor. Therefore, this testbed can produce an arbitrary
slip ratio (as defined in \cite{Slip2009}) by varying the angular speed of the
wheel and the longitudinal velocity of the unit. 

Additionally, the carriage unit can be detached from the ball screw to allow
free longitudinal movement. In this way, the wheel can be driven with free slip
recreating realistic driving conditions over sandy terrain while being able to
keep accurate monitoring of its position.

\begin{figure}
    \begin{subfigure}{.57\linewidth}
        \centering
        \input{\subdir/wheeltestbed-setup-lateral.tkiz}
        \caption{Lateral view}
        \label{fig:wheeltestbed-setup-2-lateral}
    \end{subfigure}
    \hfill
    \begin{subfigure}{.40\linewidth}
        \begin{subfigure}{\linewidth}
            \centering
            \input{\subdir/wheeltestbed-setup-back.tkiz}
            \caption{Back view}
            \label{fig:wheeltestbed-setup-2-back}
        \end{subfigure}
        \bigskip
        
        \begin{subfigure}{\linewidth}
            \centering
            \input{\subdir/wheeltestbed-setup-front.tkiz}
            \caption{Front view}
            \label{fig:wheeltestbed-setup-2-front}
        \end{subfigure}
    \end{subfigure}
    \caption{Lateral (\emph{a}), back (\emph{b}) and front (\emph{c}) view of
    the wheel test bed carriage unit. The positioning of the microphones takes
    into account ideal positions of a microphone attempting to capture the
    sound generated by the wheel interaction with the terrain (1 and 2),
    feasible positions of a microphone in a mobile robot (3, 4, and 5), and a
    reasonable position of a microphone dedicated to human-robot interaction
    (6)}
    \label{fig:wheeltestbed-setup-2}
\end{figure}

% This table is too big and irrelevant.
% \begin{table}
%     \centering
%     \begin{tabular}{| c | c|}
%         \hline
%         Index & Device                                        \\ \hline \hline
%         1     & \RealSenseDepth                               \\ \hline
%         2     & \RealSenseTracking                            \\ \hline
%         3     & \RODEVideoMicNTG{} front                      \\ \hline
%         4     & \RODEVideoMicNTG{} back                       \\ \hline
%         5     & \RODEVideoMicNTG{} top                        \\ \hline
%         6     & \RODESmartLav{} wheel axis                    \\ \hline
%         7     & \RODESmartLav{} top                           \\ \hline
%         8     & \HPEliteDragonfly{} built in microphone array \\ \hline
%     \end{tabular}
%     \caption{Devices installed in the carriage of the single wheel testbed}
%     \label{table:wheeltestbed-setup-2-devices}
% \end{table}


