\subsection{Experimental Setup} \label{subsec:experimental-setup}

This work uses a single wheel testbed, composed of a carriage unit attached to
a frame over a soil bin. It is 3500 mm in length, 600 mm in width, and 1200 mm
in height. The soil bin is filled with sandy soil with a depth of 200 mm. The
surface of the sand can be leveled or sloped before each test run using a
leveling apparatus attached to the sandbox.

The carriage unit can move in the longitudinal direction at an arbitrary
velocity using the ball screw while the unit freely moves in its vertical
direction. The wheel is placed at the bottom of the carriage wheel with an
independent traction motor. Therefore, this testbed can produce an arbitrary
slip ratio by varying the angular speed of the wheel and the longitudinal
velocity of the unit. 

The slip ratio $s$ of a smooth wheel is defined as the function of the
horizontal velocity $V_x$ and wheel angular velocity $V_\omega$ found in
\cref{eq:slip-ratio} \cite{Slip2009}, where $r$ is the radius of the wheel. The
slip ratio is bounded in the range $\left(-1,1\right]$, where negative values
correspond to wheel \emph{skidding}.

\begin{equation}
    s  = \begin{cases} 
        1 - \frac{V_x}{r V_\omega} & (r V_\omega \geq V_x, 0 \leq s \leq 1) \\
        \frac{r V_\omega}{V_x} - 1 & (r V_\omega < V_x, -1 \leq s < 0)
    \end{cases}
    \label{eq:slip-ratio}
\end{equation}

Additionally, the carriage unit can be detached from the ball screw to allow
free longitudinal movement. In this way, the wheel can be driven with free slip
recreating realistic driving conditions over sandy terrain while being able to
keep accurate monitoring of its position.

The carriage unit is equipped with a total of six microphones and two cameras,
listed in \cref{table:wheeltestbed-setup-2-devices}. Their positioning is shown
in \cref{fig:wheeltestbed-setup-2} and it  takes into account ideal positions
of a microphone attempting to capture the sound generated by the wheel
interaction with the terrain (devices 4 and 3), feasible positions of a
microphone in a mobile robot (devices 5, 6, and 7), and a reasonable position
of a microphone dedicated to human-robot interaction (device 8).

\begin{figure}
    \begin{subfigure}{.57\linewidth}
        \centering
        \input{\subdir/wheeltestbed-setup-lateral.tkiz}
        \caption{Lateral view}
        \label{fig:wheeltestbed-setup-2-lateral}
    \end{subfigure}
    \hfill
    \begin{subfigure}{.40\linewidth}
        \begin{subfigure}{\linewidth}
            \centering
            \input{\subdir/wheeltestbed-setup-back.tkiz}
            \caption{Back view}
            \label{fig:wheeltestbed-setup-2-back}
        \end{subfigure}
        \bigskip
        
        \begin{subfigure}{\linewidth}
            \centering
            \input{\subdir/wheeltestbed-setup-front.tkiz}
            \caption{Front view}
            \label{fig:wheeltestbed-setup-2-front}
        \end{subfigure}
    \end{subfigure}
    \caption{
        \nameref{subsec:experimental-setup}: \emph{a)} shows a
        lateral view of the wheel testbed carriage while being weighted.
        \emph{b)} shows a back view of the wheel and sensor setup while
        \emph{c)} shows a front view of it.}
    \label{fig:wheeltestbed-setup-2}
\end{figure}

\begin{table}
    \centering
    \begin{tabular}{| c | c|}
        \hline
        Index & Device                                        \\ \hline \hline
        1     & \RealSenseDepth                               \\ \hline
        2     & \RealSenseTracking                            \\ \hline
        3     & \RODEVideoMicNTG{} front                      \\ \hline
        4     & \RODEVideoMicNTG{} back                       \\ \hline
        5     & \RODEVideoMicNTG{} top                        \\ \hline
        6     & \RODESmartLav{} wheel axis                    \\ \hline
        7     & \RODESmartLav{} top                           \\ \hline
        8     & \HPEliteDragonfly{} built in microphone array \\ \hline
    \end{tabular}
    \caption{Devices installed in the carriage of the single wheel testbed}
    \label{table:wheeltestbed-setup-2-devices}
\end{table}


